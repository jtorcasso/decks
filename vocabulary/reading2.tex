% -*- coding: utf-8 -*-
\documentclass[12pt]{article}
\special{papersize=3in,5in}
\usepackage{amssymb,amsmath}
\pagestyle{empty}
\setlength{\parindent}{0in}
\newenvironment{note}{\paragraph{NOTE:}}{}
\newenvironment{field}{\paragraph{field:}}{}

\begin{document}

\begin{note}
\begin{field}
\textbf{\large cataract}
\end{field}

\xplain{cataract_reading2.jpg}

\begin{field}
\begin{description}
\item[noun] \hfill \\ 
an eye disease that involves the clouding or opacification of the natural lens of the eye

\item[noun] \hfill \\ 
a large waterfall; violent rush of water over a precipice

\end{description}
\end{field}

\begin{field}
\begin{itemize}
\item 
\item There are no drops or tablets that will treat cataracts
\item Cataract extraction
\item Senile cataracts almost always occur in both eyes simultaneously
\item Cats tend to develop cataracts at an older age than dogs
\end{itemize}
\end{field}
\end{note}
\begin{note}
\begin{field}
\textbf{\large inexorably}
\end{field}

\xplain{inexorably_reading2.jpg}

\begin{field}
\begin{description}
\item[adv] \hfill \\ 
in an inexorable manner

\end{description}
\end{field}

\begin{field}
\end{field}
\end{note}
\begin{note}
\begin{field}
\textbf{\large baronial}
\end{field}

\xplain{baronial_reading2.jpg}

\begin{field}
\begin{description}
\item[adj] \hfill \\ 
impressive in appearance

\end{description}
\end{field}

\begin{field}
\begin{itemize}
\item 
\item Baronial architecture
\item Baronial residence for over five centuries
\item Burn was a great exponent of the Scottish baronial style and encased the whole ancient edifice within a baronial mansion
\item Baronial splendor of The Fairmont Banff Springs never fails to amaze
\end{itemize}
\end{field}
\end{note}
\begin{note}
\begin{field}
\textbf{\large phillistine}
\end{field}

\xplain{phillistine_reading2.jpg}

\begin{field}
\end{field}

\begin{field}
\end{field}
\end{note}
\begin{note}
\begin{field}
\textbf{\large recalcitrant}
\end{field}

\xplain{recalcitrant_reading2.jpg}

\begin{field}
\begin{description}
\item[adj] \hfill \\ 
stubbornly resistant to authority or control

\item[adj] \hfill \\ 
marked by stubborn resistance to authority

\end{description}
\end{field}

\begin{field}
\begin{itemize}
\item One particularly recalcitrant problem is described in Menzies ( 1989 )
\item 
\item Recalcitrant child 's homework
\item Recalcitrant party
\item Recalcitrant states can claim the dubious honor of having failed to ratify either convention
\end{itemize}
\end{field}
\end{note}
\begin{note}
\begin{field}
\textbf{\large seignorial}
\end{field}

\xplain{seignorial_reading2.jpg}

\begin{field}
\end{field}

\begin{field}
\end{field}
\end{note}
\begin{note}
\begin{field}
\textbf{\large endogamy}
\end{field}

\xplain{endogamy_reading2.jpg}

\begin{field}
\begin{description}
\item[noun] \hfill \\ 
marriage within one's own tribe or group as required by custom or law

\end{description}
\end{field}

\begin{field}
\end{field}
\end{note}
\begin{note}
\begin{field}
\textbf{\large miscegenation}
\end{field}

\xplain{miscegenation_reading2.jpg}

\begin{field}
\begin{description}
\item[noun] \hfill \\ 
reproduction by parents of different races (especially by white and non-white persons)

\end{description}
\end{field}

\begin{field}
\begin{itemize}
\item Miscegenation of state, commercial and community interests within the emerging digital economy
\item They promote musical miscegenation, drug use and wanton sexual license
\item They promote musical miscegenation, drug use and wanton sexual license
\item 
\end{itemize}
\end{field}
\end{note}
\begin{note}
\begin{field}
\textbf{\large putative}
\end{field}

\xplain{putative_reading2.jpg}

\begin{field}
\begin{description}
\item[adj] \hfill \\ 
purported; commonly put forth or accepted as true on inconclusive grounds

\end{description}
\end{field}

\begin{field}
\begin{itemize}
\item Putative father has died
\item 
\item Putative hybrid between these two
\item Putative insurer is
\item Putative store-operated calcium channel genes
\end{itemize}
\end{field}
\end{note}
\begin{note}
\begin{field}
\textbf{\large syncretist}
\end{field}

\xplain{syncretist_reading2.jpg}

\begin{field}
\end{field}

\begin{field}
\end{field}
\end{note}
\begin{note}
\begin{field}
\textbf{\large intransigent}
\end{field}

\xplain{intransigent_reading2.jpg}

\begin{field}
\begin{description}
\item[adj] \hfill \\ 
impervious to pleas, persuasion, requests, reason; she would have none of him"- W.Churchill

\end{description}
\end{field}

\begin{field}
\begin{itemize}
\item 
\item Intransigent position
\item This could result in further industrial action in May if the company remains intransigent
\item I point out that I am doing it as a public service, and earning nothing from it, but they remain intransigent
\item BT is trying to create business links with the music industry at the same time as being completely intransigent to the issues of piracy
\end{itemize}
\end{field}
\end{note}
\begin{note}
\begin{field}
\textbf{\large ultramontane}
\end{field}

\xplain{ultramontane_reading2.jpg}

\begin{field}
\begin{description}
\item[noun] \hfill \\ 
a Roman Catholic who advocates ultramontanism (supreme papal authority in matters of faith and discipline)

\item[adj] \hfill \\ 
of or relating to ultramontanism

\item[adj] \hfill \\ 
on or relating to or characteristic of the region or peoples beyond the Alps from Italy (or north of the Alps)

\item[adj] \hfill \\ 
on the Italian or Roman side of the Alps

\end{description}
\end{field}

\begin{field}
\end{field}
\end{note}
\begin{note}
\begin{field}
\textbf{\large execrate}
\end{field}

\xplain{execrate_reading2.jpg}

\begin{field}
\begin{description}
\item[verb] \hfill \\ 
find repugnant

\item[verb] \hfill \\ 
curse or declare to be evil or anathema or threaten with divine punishment

\end{description}
\end{field}

\begin{field}
\end{field}
\end{note}
\begin{note}
\begin{field}
\textbf{\large annuated}
\end{field}

\xplain{annuated_reading2.jpg}

\begin{field}
\end{field}

\begin{field}
\end{field}
\end{note}
\begin{note}
\begin{field}
\textbf{\large brigandage}
\end{field}

\xplain{brigandage_reading2.jpg}

\begin{field}
\end{field}

\begin{field}
\end{field}
\end{note}
\begin{note}
\begin{field}
\textbf{\large esoteric}
\end{field}

\xplain{esoteric_reading2.jpg}

\begin{field}
\begin{description}
\item[adj] \hfill \\ 
confined to and understandable by only an enlightened inner circle

\end{description}
\end{field}

\begin{field}
\begin{itemize}
\item 
\item He was avoiding the sense of making something too esoteric
\item It boardered on esoteric doctrine
\item What follows are the more advanced, less used, or sometimes esoteric capabilities of perl regexps
\item Our only concerns are the Canon menu system, which is just too esoteric for our liking
\end{itemize}
\end{field}
\end{note}

\end{document}